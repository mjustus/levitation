\documentclass{article}

\usepackage{amsmath}
\usepackage{amssymb}
\usepackage[utf8]{inputenc}
\usepackage{fontenc}
\usepackage{stmaryrd}

\usepackage{pdflscape}

\usepackage{tabularx}
\newcolumntype{Y}{>{\centering\arraybackslash}X}

\usepackage{mathpartir}
\renewcommand{\TirNameStyle}[1]{{\small #1}}

\usepackage{url}
\usepackage{hyperref}

\usepackage{color}

\definecolor{term}{RGB}{153,0,0}
\definecolor{type}{RGB}{0,0,224}
\definecolor{definition}{RGB}{0,102,0}
\definecolor{neutral}{RGB}{102,0,102}

\usepackage[top=0.75in, bottom=1in, left=1in, right=1in]{geometry}

\title{Levitation without the sugar}

\newcommand{\ENT}{\vdash}
\newcommand{\OF}{:}
\newcommand{\EQ}{\ensuremath{\mathbin{\equiv}}}

\newcommand{\type}[1]{{\textcolor{type}{\textsf{#1}}}}
\newcommand{\term}[1]{{\textcolor{term}{\textsf{#1}}}}
\newcommand{\definition}[1]{{\textcolor{definition}{\textsf{#1}}}}
\newcommand{\unv}[1]{{\textcolor{type}{\textsc{#1}}}}

\newcommand{\name}[1]{{\textcolor{neutral}{#1}}}

\newcommand{\Set}{\unv{Set}}
\newcommand{\One}{\type{1}}
\newcommand{\To}{\mathbin{\textcolor{type}{\shortrightarrow}}}
\newcommand{\App}{\,}
\newcommand{\lam}[2]{{\textcolor{term}{\lambda}}_{#1} {\name #2}.\,}

\newcommand{\mult}{\mathbin{\textcolor{type}{\times}}}
\newcommand{\proj}[1]{\mathop{\textcolor{definition}{\pi_{#1}}}}
\newcommand{\cons}[2]{\textcolor{term}{[}#1\textcolor{term}{,}\, #2\textcolor{term}{]}}
\newcommand{\nil}{\textcolor{term}{[]}}

\newcommand{\SigmaT}[3]{\textcolor{type}{(}\name #1 \OF #2 \textcolor{type}{)} \mult #3}
\newcommand{\PiT}[3]{\textcolor{type}{(}\name #1 \OF #2 \textcolor{type}{)} \To #3}

\newcommand{\plus}{\mathbin{\textcolor{type}{+}}}
\newcommand{\inj}[1]{\mathop{\textcolor{term}{\iota_{#1}}}}
\newcommand{\match}[2]{\textcolor{definition}{\langle}#1\textcolor{definition}{,}\, #2\textcolor{definition}{\rangle}}


\newcommand{\Tag}{\type{Tag}}
\newcommand{\quot}{\textcolor{term}{`}}

\newcommand{\En}{\type{En}}
\newcommand{\nilE}{\term{nE}}
\newcommand{\consE}[2]{\term{cE} \App {#1} \App {#2}}

\newcommand{\hash}{\type{\#}}
\newcommand{\zero}{\term{0}}
\newcommand{\suc}{\term{1+}}

\newcommand{\spi}[2]{\textcolor{definition}{\pi} \App {#1} \App {#2}}
\newcommand{\switch}[4]{\definition{switch} \App {#1} \App {#2}  \App {#3}  \App {#4}}
\newcommand{\spiD}[1]{\definition{$\pi$D} \App {#1}}
\newcommand{\switchD}[3]{\definition{switchD} \App {#1} \App {#2}  \App {#3}}

\newcommand{\Desc}{\type{Desc}}

\newcommand{\dOne}{\quot\textcolor{term}{1}}
\newcommand{\dSigma}{\quot\textcolor{term}{\Sigma}}
\newcommand{\dInd}{\quot\textcolor{term}{\textsf{indx}}}
\newcommand{\dHind}{\quot\textcolor{term}{\textsf{hindx}}}

\newcommand{\cOne}{\con \App \cons \dOne \nil}
\newcommand{\cSigma}[2]{\con \App \cons \dSigma {\cons {#1}  {\cons {#2} \nil}}}
\newcommand{\cInd}[1]{\con \App \cons \dInd {\cons #1 \nil}}
\newcommand{\cHind}[2]{\con \App \cons \dHind {\cons #1  {\cons #2 \nil}}}

\newcommand{\uOne}{\underline{\dOne}}
\newcommand{\uSigma}[2]{\underline{\dSigma} \App #1 \App #2}
\newcommand{\uInd}[1]{\underline{\dInd} \App #1}
\newcommand{\uHind}[2]{\underline{\dHind} \App #1 \App #2}

\newcommand{\interp}[2]{\textcolor{definition}{\llbracket} #1 \textcolor{definition}{\rrbracket} \App #2}

\newcommand{\fix}{\textcolor{type}{\mu}}
\newcommand{\con}{\term{con}}
\newcommand{\ind}[3]{\definition{ind} \App {#1} \App {#2}  \App {#3}}
\newcommand{\iAll}[4]{\definition{All} \App {#1} \App {#2}  \App {#3} \App {#4}}
\newcommand{\iall}[4]{\definition{all} \App {#1} \App {#2}  \App {#3} \App {#4}}

\usepackage[backend=bibtex, firstinits=true]{biblatex}
\begin{filecontents}{levitation.bib}
@Book{Martin-Loef1984,
  Title                    = {Intuitionistic Type Theory: Notes by {G}iovanni {S}ambin of a series of lectures given in {P}adua, {J}une 1980},
  Author                   = {Martin-L\"of, Per},
  Publisher                = {Bibliopolis},
  Year                     = {1984},
  Series                   = {Studies in Proof Theory}
}

@inproceedings{levitation,
 author = {Chapman, James and Dagand, Pierre-\'{E}variste and McBride, Conor and Morris, Peter},
 title = {The Gentle Art of Levitation},
 booktitle = {Proceedings of the 15th ACM SIGPLAN International Conference on Functional Programming},
 series = {ICFP '10},
 year = {2010},
 pages = {3--14}
} 

\end{filecontents}
\bibliography{levitation}

\begin{document}

\maketitle

We present the low-level type theory of ``The Gentle Art of
Levitation''~\cite{levitation} in a more familiar rule-based style, organised
as much as possible into the usual categories of formation, introduction,
etc. The only real (but inconsequential) difference is that we require type
constructor to always be fully applied. Otherwise we follow the paper closely,
including the use of colour\footnote{See
  \url{http://strictlypositive.org/dpm/colour.html} for the rationale behind
  the choice of colours.} to distinguish the different categories of
syntax. In brief, \type{type formers} are blue, \term{constructors} red,
\definition{defined functions} green, \name{bound variables} purple, and
meta-variables stay uncoloured.


\section{Standard}

We recall the standard rules for unit- and $\Sigma$-types,
mainly to introduce the slightly non-standard syntax.

\subsection{Unit type}

\subsubsection*{Formation}
\begin{mathpar}
\inferrule*{\Gamma \ENT {}} {
  \Gamma \ENT \One \OF \Set
}
\end{mathpar}

\subsubsection*{Introduction}
\begin{mathpar}
\inferrule*{\Gamma \ENT {}} {
  \Gamma \ENT \nil \OF \One
}
\end{mathpar}

\subsection{$\Sigma$-types}

\subsubsection*{Formation}
\begin{mathpar}
\inferrule*{
  \Gamma \ENT S \OF \Set\\
  \Gamma; {\name x \OF S} \ENT T \OF \Set
} {
  \Gamma \ENT \SigmaT x S T \OF \Set
}
\end{mathpar}

\subsubsection*{Introduction}
\begin{mathpar}
\inferrule*{
  \Gamma \ENT S \OF \Set\\
  \Gamma; {\name x \OF S} \ENT T \OF \Set
} {
  \Gamma \ENT {\cons s t}_{{\name x. T}} \OF \SigmaT x S T
}
\end{mathpar}
To save space, we will drop the type annotation when it is clear from the
context. We will also write
\begin{equation*}
\left[
\begin{array}{l}
t_1\\
\ldots\\
t_n
\end{array}
\right]
\end{equation*}
for the right-nested pair $\cons {t_1} {\cons {\ldots} {{\cons {t_n} \nil} \ldots}}$.

\subsubsection*{Elimination}
\begin{mathpar}
\inferrule*{
  \Gamma \ENT p \OF \SigmaT x S T
} {
  \Gamma \ENT \proj 0 \App p \OF S
}

\inferrule*{
  \Gamma \ENT p \OF \SigmaT x S T
} {
  \Gamma \ENT \proj 1 \App p \OF T [\proj 0 \App p/\name x]
}
\end{mathpar}

\subsubsection*{Computation}
\begin{mathpar}
\inferrule*{
  \Gamma \ENT s \OF S\\
  \Gamma \ENT t \OF T [s/\name x]
} {
  \Gamma \ENT {
    \proj 0 \App {\cons s t}_{\name x. T}
  } \EQ {
    s
  } \OF S
}

\inferrule*{
  \Gamma \ENT s \OF S\\
  \Gamma \ENT t \OF T [s/\name x]
} {
  \Gamma \ENT {
    \proj 1 \App {\cons s t}_{\name x. T}
  } \EQ {
    t
  } \OF T [s/\name x]
}
\end{mathpar}

\section{Finite sets and small products}

\subsection{Tags}
The purpose of tags is to provide the user with identifiers to be used as
constructor names. As this is their only purpose, they only come with an
introduction but not elimination rule.

\subsubsection*{Formation}
\begin{mathpar}
\inferrule*{\Gamma \ENT {}} {
  \Gamma \ENT \Tag \OF \Set
}
\end{mathpar}

\subsubsection*{Introduction}
\begin{mathpar}
\inferrule*[Right=$\name s$ is a valid identifier]{\Gamma \ENT {}} {
  \Gamma \ENT \quot \name s \OF \Set
}
\end{mathpar}
The paper leaves open what it means for an identifier to be ``valid''.


\begin{landscape}
\subsection{Small products indexed by \En}

\subsubsection*{\En-formation}

\begin{tabularx}{\linewidth}{|*{3}{Y|}}
\begin{mathpar}
  \inferrule*{\Gamma \ENT {}} {
    \Gamma \ENT \En \OF \Set
  }
\end{mathpar}
&
\begin{mathpar}
\inferrule{\Gamma \ENT E \OF \En} {
  \Gamma \ENT \hash E \OF \Set
}
\end{mathpar}
&
\begin{mathpar}
\inferrule{
  \Gamma \ENT E \OF \En\\
  \Gamma \ENT P \OF \hash E \To \Set
} {
  \Gamma \ENT \spi E P \OF \Set
}
\end{mathpar}
\end{tabularx}

\subsubsection*{\En-introduction}
\begin{tabularx}{\linewidth}{|*{3}{Y|}}
\begin{mathpar}
\inferrule*{\Gamma \ENT {}} {
  \Gamma \ENT \nilE \OF \En
}
\end{mathpar}
&
&
\begin{mathpar}
\inferrule*{
  \Gamma \ENT P \OF \hash \nilE \To \Set
} {
  \Gamma \ENT {\spi \nilE P} \EQ \One \OF \Set
}
\end{mathpar}
\\
\begin{mathpar}
\inferrule*{
  \Gamma \ENT t \OF \Tag \\
  \Gamma \ENT E \OF \En
} {
  \Gamma \ENT \consE t E \OF \En
}
\end{mathpar}
&
\begin{mathpar}
\inferrule*{
  \Gamma \ENT t \OF \Tag\\
  \Gamma \ENT E \OF \En
} {
  \Gamma \ENT \zero \OF \hash (\consE t E)
}

\inferrule*{
  \Gamma \ENT n \OF \hash E
} {
  \Gamma \ENT \suc n \OF \hash (\consE t E)
}
\end{mathpar}
&
\begin{mathpar}
\inferrule*{
  \Gamma \ENT t \OF \Tag\\\\
  \Gamma \ENT E \OF \En\\\\
  \Gamma \ENT P \OF \hash (\consE t E) \To \Set
} {
  \Gamma \ENT {
    \spi {(\consE t E)} P
  } \EQ {
    P \App \zero \mult \spi E {(\lam {\hash E} x P \App (\suc \name x))}
  } \OF \Set
}
\end{mathpar}
\end{tabularx}

\subsubsection*{\En-elimination}
\begin{mathpar}
\inferrule*{
  \Gamma \ENT E \OF \En\\
  \Gamma \ENT P \OF \hash E \To \Set\\
  \Gamma \ENT b \OF \spi E P\\
  \Gamma \ENT x \OF \hash E
} {
  \Gamma \ENT \switch E P b x \OF P \App x
}
\end{mathpar}

\subsubsection*{\En-computation}
\begin{mathpar}
\inferrule*{
  \Gamma \ENT t \OF \Tag\\
  \Gamma \ENT E \OF \En\\
  \Gamma \ENT P \OF \hash (\consE t E) \To \Set\\\\
  \Gamma \ENT b \OF P \App \zero \mult \spi E {(\lam {\hash E} x P \App (\suc \name x))}
} {
  \Gamma \ENT {
    \switch {(\consE t E)} P b \zero
   } \EQ {
     \proj 0 b
   } \OF P \App \zero
}

\inferrule*{
  \Gamma \ENT t \OF \Tag\\
  \Gamma \ENT E \OF \En\\
  \Gamma \ENT P \OF \hash (\consE t E) \To \Set\\\\
  \Gamma \ENT b \OF P \App \zero \mult \spi E {(\lam {\hash E} x P \App (\suc \name x))}\\\\
  \Gamma \ENT x \OF \hash E
} {
  \Gamma \ENT {
    \switch {(\consE t E)} P b {(\suc x)}
   } \EQ {
     \switch E {(\lam {\hash E} x P \App (\suc \name x))} {(\proj 1 b)} x
   } \OF P (\suc \name x)
}
\end{mathpar}
\end{landscape}


\section{Descriptions and their interpretation}

\subsection{Descriptions}

\subsubsection*{\Desc-formation}
Again, this is Tarski-style universe construction.

\noindent\begin{tabularx}{\linewidth}{|*{2}{Y|}}
\begin{mathpar}
\inferrule* {
  \Gamma \ENT {}
} {
  \Gamma \ENT \Desc \OF \Set
}
\end{mathpar}
&
\begin{mathpar}
\inferrule* {
  \Gamma \ENT D \OF \Desc\\
  \Gamma \ENT X \OF \Set
} {
  \Gamma \ENT \interp D X \OF \Set
}
\end{mathpar}
\end{tabularx}

\subsubsection*{\Desc-introduction}
We do need any rules introducing \Desc-codes as they will be generated by
levitation (see below). The resulting constructors are listed on the left-hand
side together with suitable abbreviations, which are for our convenience only
and not part of the type theory.

\noindent\begin{tabularx}{\linewidth}{|*{2}{Y|}}
\begin{mathpar}
\uOne := \cOne
\end{mathpar}
&
\begin{mathpar}
\inferrule* {
  \Gamma \ENT X \OF \Set
} {
  \Gamma \ENT {\interp \uOne X} \EQ \One \OF \Set
}
\end{mathpar}
\\
\begin{mathpar}
\uSigma S D := \cSigma S D
\end{mathpar}
&
\begin{mathpar}
\inferrule* {
  \Gamma \ENT X \OF \Set\\
  \Gamma \ENT D \OF S \To \Desc\\
  \Gamma \ENT S \OF \Set
} {
  \Gamma \ENT {
    \interp {\uSigma S D} X
  } \EQ {
    \SigmaT s S {\interp {D \App \name s} X}
  } \OF \Set
}
\end{mathpar}
\\
\begin{mathpar}
\uInd D := \cInd D
\end{mathpar}
&
\begin{mathpar}
\inferrule* {
  \Gamma \ENT X \OF \Set\\
  \Gamma \ENT D \OF \Desc
} {
  \Gamma \ENT {
    \interp {\uInd D} X
  } \EQ {
    X \mult {\interp D X}
  } \OF \Set
}
\end{mathpar}
\\
\begin{mathpar}
\uHind H D := \cHind H D
\end{mathpar}
&
\begin{mathpar}
\inferrule* {
  \Gamma \ENT X \OF \Set\\
  \Gamma \ENT H \OF \Set\\
  \Gamma \ENT D \OF \Desc
} {
  \Gamma \ENT {
    \interp {\uHind H D} X
  } \EQ {
    (H \To X) \mult {\interp D X}
  } \OF \Set
}
\end{mathpar}
\end{tabularx}

\subsubsection*{\Desc-levitation}
\begin{mathpar}
\inferrule* {
  \Gamma \ENT {}
} {
  \Gamma \ENT
    \Desc
  \EQ {
    \fix {
      \left(\uSigma
        {\hash \underline {\term E}}
        {\left(\lam {\hash \underline {\term E}} x \switchD {\underline {\term E}}
{\left[
\begin{array}{l}
\uOne\\
\uSigma \Set {(\lam \Set S \uHind {\name S} \uOne)}\\
\uInd \uOne\\
\uSigma {\Set} {(\lam \Set \_ \uInd \uOne)}
\end{array}
\right]}
{\name x}\right)}
      \right)
    }
  } \OF \Set
}
\end{mathpar}
where $\underline {\term E}$ is the enumeration $\consE \dOne {(\consE \dSigma {(\consE \dInd {(\consE \dHind \nilE)})})}$.


%%%
% switchD using spi
%%%
\subsection{Eliminating small products of descriptions}

\subsubsection*{Elimination}
\begin{mathpar}
\inferrule*{
  \Gamma \ENT E \OF \En\\
  \Gamma \ENT b \OF \spi E {(\lam {\hash E} \_ \Desc)}\\
  \Gamma \ENT x \OF \hash E
} {
  \Gamma \ENT \switchD E b x \OF \Desc
}
\end{mathpar}

\subsubsection*{Computation}
\begin{mathpar}
\inferrule*{
  \Gamma \ENT t \OF \Tag\\
  \Gamma \ENT E \OF \En\\
  \Gamma \ENT b \OF \Desc \mult \spi E {(\lam {\hash E} \_ \Desc)}
} {
  \Gamma \ENT {
    \switchD {(\consE t E)} b \zero
   } \EQ {
     \proj 0 b
   } \OF P \App \zero
}

\inferrule*{
  \Gamma \ENT t \OF \Tag\\
  \Gamma \ENT E \OF \En\\
  \Gamma \ENT b \OF \Desc \mult \spi E {(\lam {\hash E} \_ \Desc)}\\
  \Gamma \ENT x \OF \hash E
} {
  \Gamma \ENT {
    \switchD {(\consE t E)} b {(\suc x)}
   } \EQ {
     \switchD E {(\proj 1 b)} x
   } \OF \Desc
}
\end{mathpar}

\subsection{Fixpoint}

\subsubsection*{Formation}
\begin{mathpar}
\inferrule* {
  \Gamma \ENT D \OF \Desc
} {
  \Gamma \ENT \fix D \OF \Set
}
\end{mathpar}

\subsubsection*{Introduction}
\begin{mathpar}
\inferrule* {
  \Gamma \ENT D \OF \Desc\\
  \Gamma \ENT d \OF \interp D {(\fix D)}
} {
  \Gamma \ENT \con \App d \OF \fix D
}
\end{mathpar}

\subsubsection*{Elimination}
\begin{mathpar}
\inferrule*{
  \Gamma \ENT D \OF \Desc\\\\
  \Gamma \ENT P \OF \fix D \To \Set\\\\
  \Gamma \ENT m \OF \textcolor{type}{(} \name d \OF \interp {D} {(\fix D)} \textcolor{type}{)} \To \iAll D {(\fix D)} P {\name d} \To P \App (\con \App {\name d})\\\\
  \Gamma \ENT x \OF \fix D
} {
  \Gamma \ENT \ind D P m \App x \OF P \App x
}
\end{mathpar}

\subsubsection*{Computation}
\begin{mathpar}
\inferrule*{
  \Gamma \ENT D \OF \Desc\\\\
  \Gamma \ENT P \OF \fix D \To \Set\\\\
  \Gamma \ENT m \OF \textcolor{type}{(} \name d \OF \interp {D} {(\fix D)} \textcolor{type}{)} \To \iAll D {(\fix D)} P {\name d} \To P \App (\con \App {\name d})\\\\
  \Gamma \ENT d \OF \interp D {(\fix D)}
} {
  \Gamma \ENT {
    \ind D P m \App {(\con \App d)}
  } \EQ {
    m \App d \App (\iall D {(\fix D)} P {(\ind D P m)} \App d)
  } \OF P \App x
}
\end{mathpar}

\begin{landscape}
\subsection{\definition{All} predicate}

\subsubsection*{Formation}
\begin{mathpar}
\inferrule*{
  \Gamma \ENT D \OF \Desc\\
  \Gamma \ENT X \OF \Set\\
  \Gamma \ENT P \OF X \To \Set\\
  \Gamma \ENT xs \OF \interp D X
} {
  \Gamma \ENT \iAll D X P xs \OF \Set
}
\end{mathpar}

\subsubsection*{Introduction}
\begin{tabularx}{\linewidth}{|*{2}{Y|}}
\begin{mathpar}
\inferrule*{
  \Gamma \ENT X \OF \Set\\\\
  \Gamma \ENT P \OF X \To \Set
} {
  \Gamma \ENT {
    \iAll \uOne X P {\nil}
  } \EQ \One \OF \Set
}
\end{mathpar}
&
\begin{mathpar}
\inferrule*{
  \Gamma \ENT X \OF \Set\\\\
  \Gamma \ENT P \OF X \To \Set\\
  \Gamma \ENT p \OF \PiT x X {P \App {\name x}}
} {
  \Gamma \ENT {
    \iall \uOne X P p \, \nil
  } \EQ \nil \OF \One
}
\end{mathpar}
\\
\begin{mathpar}
\inferrule*{
  \Gamma \ENT S \OF \Set\\
  \Gamma \ENT D \OF S \To \Desc\\
  \Gamma \ENT X \OF \Set\\\\
  \Gamma \ENT P \OF X \To \Set
} {
  \Gamma \ENT {
    \iAll {(\uSigma S D)} X P {\cons s d}
  } \EQ {
    \iAll {(D \, s)} X P d
  } \OF \Set
}
\end{mathpar}
&
\begin{mathpar}
\inferrule*{
  \Gamma \ENT S \OF \Set\\
  \Gamma \ENT D \OF S \To \Desc\\
  \Gamma \ENT X \OF \Set\\\\
  \Gamma \ENT P \OF X \To \Set\\
  \Gamma \ENT p \OF \PiT x X {P \App {\name x}}\\\\
  \Gamma \ENT s \OF S\\
  \Gamma \ENT d \OF D \App s
} {
  \Gamma \ENT {
    \iall {(\uSigma S D)} X P p \App {\cons s d}
  } \EQ {
    \iall {(D \, s)} X P p \App d
  } \OF \iAll {(D \, s)} X P d
}
\end{mathpar}
\\
\begin{mathpar}
\inferrule*{
  \Gamma \ENT D \OF \Desc\\
  \Gamma \ENT X \OF \Set\\\\
  \Gamma \ENT P \OF X \To \Set
} {
  \Gamma \ENT {
    \iAll {(\uInd D)} X P {\cons x d}
  } \EQ {
    P \App x \mult \iAll D X P d
  } \OF \Set
}
\end{mathpar}
&
\begin{mathpar}
\inferrule*{
  \Gamma \ENT D \OF \Desc\\
  \Gamma \ENT X \OF \Set\\\\
  \Gamma \ENT P \OF X \To \Set\\
  \Gamma \ENT p \OF \PiT x X {P \App {\name x}}\\\\
  \Gamma \ENT x \OF X\\
  \Gamma \ENT d \OF D
} {
  \Gamma \ENT {
    \iall {(\uInd D)} X P p \App {\cons x d}
  } \EQ {
    \cons {p \App x} {\iall D X P p \App d}
  } \OF P \App x \mult \iAll D X P d
}
\end{mathpar}
\\
\begin{mathpar}
\inferrule*{
  \Gamma \ENT H \OF \Set\\
  \Gamma \ENT D \OF \Desc\\
  \Gamma \ENT X \OF \Set\\\\
  \Gamma \ENT P \OF X \To \Set
} {
  \Gamma \ENT {
    \iAll {(\uHind H D)} X P {\cons f d}
  } \EQ {
    (\PiT h H {P \App (f \App {\name h})}) \mult \iAll D X P d
  } \OF \Set
}
\end{mathpar}
&
\begin{mathpar}
\inferrule*{
  \Gamma \ENT H \OF \Set\\
  \Gamma \ENT D \OF \Desc\\
  \Gamma \ENT X \OF \Set\\\\
  \Gamma \ENT P \OF X \To \Set\\
  \Gamma \ENT p \OF \PiT x X {P \App {\name x}}\\\\
  \Gamma \ENT f \OF H \To X\\
  \Gamma \ENT d \OF D
} {
  \Gamma \ENT {
    \iall {(\uHind H D)} X P p \App {\cons f d}
  } \EQ {
    \cons {\lam H h {p \App (f \App {\name h})}} {\iall D X P p \App d}
  } \OF (\PiT h H {P \App (f \App {\name h})}) \mult \iAll D X P d
}
\end{mathpar}
\end{tabularx}
\end{landscape}

\printbibliography

\appendix

%%%
% \En using switchD only
%%%
\section{Small products of descriptions}
We spent a lot of effort to define small products over arbitrary types, only
to then introduce a special-cased projection function, \definition{switchD}, restricted to
products of descriptions. Below we give an alternative definition of
\definition{switchD} using small products of descriptions from the start.

\subsubsection*{\En-formation}

\begin{tabularx}{\linewidth}{|*{3}{Y|}}
\begin{mathpar}
  \inferrule*{\Gamma \ENT {}} {
    \Gamma \ENT \En \OF \Set
  }
\end{mathpar}
&
\begin{mathpar}
\inferrule{\Gamma \ENT E \OF \En} {
  \Gamma \ENT \hash E \OF \Set
}
\end{mathpar}
&
\begin{mathpar}
\inferrule{
  \Gamma \ENT E \OF \En
} {
  \Gamma \ENT \spiD E \OF \Set
}
\end{mathpar}
\end{tabularx}

\subsubsection*{\En-introduction}
\begin{tabularx}{\linewidth}{|*{3}{Y|}}
\begin{mathpar}
\inferrule*{\Gamma \ENT {}} {
  \Gamma \ENT \nilE \OF \En
}
\end{mathpar}
&
&
\begin{mathpar}
\inferrule*{ } {
  \Gamma \ENT {\spiD \nilE} \EQ \One \OF \Set
}
\end{mathpar}
\\
\begin{mathpar}
\inferrule*{
  \Gamma \ENT t \OF \Tag \\
  \Gamma \ENT E \OF \En
} {
  \Gamma \ENT \consE t E \OF \En
}
\end{mathpar}
&
\begin{mathpar}
\inferrule*{
  \Gamma \ENT t \OF \Tag\\
  \Gamma \ENT E \OF \En
} {
  \Gamma \ENT \zero \OF \hash (\consE t E)
}

\inferrule*{
  \Gamma \ENT n \OF \hash E
} {
  \Gamma \ENT \suc n \OF \hash (\consE t E)
}
\end{mathpar}
&
\begin{mathpar}
\inferrule*{
  \Gamma \ENT t \OF \Tag\\\\
  \Gamma \ENT E \OF \En
} {
  \Gamma \ENT {
    \spiD {(\consE t E)}
  } \EQ {
    \Desc \mult \spiD E
  } \OF \Set
}
\end{mathpar}
\end{tabularx}

\subsubsection*{\En-elimination}
\begin{mathpar}
\inferrule*{
  \Gamma \ENT E \OF \En\\
  \Gamma \ENT b \OF \spiD E\\
  \Gamma \ENT x \OF \hash E
} {
  \Gamma \ENT \switchD E b x \OF \Desc
}
\end{mathpar}

\subsubsection*{\En-computation}
\begin{mathpar}
\inferrule*{
  \Gamma \ENT t \OF \Tag\\
  \Gamma \ENT E \OF \En\\
  \Gamma \ENT b \OF \Desc \mult \spiD E
} {
  \Gamma \ENT {
    \switchD {(\consE t E)} b \zero
   } \EQ {
     \proj 0 b
   } \OF P \App \zero
}

\inferrule*{
  \Gamma \ENT t \OF \Tag\\
  \Gamma \ENT E \OF \En\\
  \Gamma \ENT b \OF \Desc \mult \spiD E\\
  \Gamma \ENT x \OF \hash E
} {
  \Gamma \ENT {
    \switchD {(\consE t E)} b {(\suc x)}
   } \EQ {
     \switchD E {(\proj 1 b)} x
   } \OF \Desc
}
\end{mathpar}

\section{Is {\En} really necessary?}

Given a type with exactly four inhabitants, it ought to be possible to
eliminate the use of {\En} in the levitation rule defining \Desc, using each
of the four inhabitants to name one constructor of \Desc.

To encode such a type, we add binary sums to the type system. Another approach
would be to introduce a {\type {Bool}} type and encode binary sums as
$A \plus B \triangleq \SigmaT b {\type{Bool}} \term{if} ~ {\name b} \term ~
\term{then} ~ A ~ \term{else} ~ B$. Yet another would be to add a four-element
type as a primitive.

\subsection{$\textcolor{blue}{+}$-types}

\subsubsection*{Formation}
\begin{mathpar}
\inferrule*{
  \Gamma \ENT S \OF \Set\\
  \Gamma \ENT T \OF \Set
} {
  \Gamma \ENT S \plus T  \OF \Set
}
\end{mathpar}

\subsubsection*{Introduction}
\begin{mathpar}
\inferrule*{
  \Gamma \ENT s \OF S\\
  \Gamma \ENT T \OF \Set 
} {
  \Gamma \ENT \inj 0 s \OF S \plus T
}

\inferrule*{
  \Gamma \ENT S \OF \Set\\
  \Gamma \ENT t \OF T
} {
  \Gamma \ENT \inj 1 s \OF S \plus T
}
\end{mathpar}

\subsubsection*{Elimination}
\begin{mathpar}
\inferrule*{
  \Gamma \ENT f \OF S \To X\\
  \Gamma \ENT g \OF T \To X\\
  \Gamma \ENT e \OF S \plus T\\
} {
  \Gamma \ENT \match f  g \, e \OF X
}
\end{mathpar}

\subsubsection*{Computation}
\begin{mathpar}
\inferrule*{
  \Gamma \ENT f \OF S \To X\\
  \Gamma \ENT g \OF T \To X\\
  \Gamma \ENT s \OF S
} {
  \Gamma \ENT {
    \match f g \App {(\inj 0 s)}
  } \EQ {
    f \App s
  } \OF X
}

\inferrule*{
  \Gamma \ENT f \OF S \To X\\
  \Gamma \ENT g \OF T \To X\\
  \Gamma \ENT t \OF T
} {
  \Gamma \ENT {
    \match f g \App {(\inj 1 t)}
  } \EQ {
    g \App t
  } \OF X
}
\end{mathpar}

\subsection{A type with exactly four inhabitants}

Let $\underline {\type 4}$ be the type give by $\One \plus {(\One \plus {(\One \plus \One)})}$. Its only inhabitants are
\begin{align*}
\underline{\term 0} &:= \inj 0 \App \nil\\
\underline{\term 1} &:= \inj 1 \App (\inj 0 \App \nil)\\
\underline{\term 2} &:= \inj 1 \App (\inj 1 \App (\inj 0 \App \nil))\\
\underline{\term 3} &:= \inj 1 \App (\inj 1 \App (\inj 1 \App (\inj 0 \App \nil)))
\end{align*}
Furthermore, we can define an eliminator
\begin{equation*}
\underline {\definition{switch4}} \App a \App b \App c \App d \App x :=
\match
  {\lam \One \_ a}
  {\lam
    {\One\plus\One\plus\One} x
    {\match
      {\lam \One \_ b}
      {\lam {\One\plus\One} x
        {\match {{\lam \One \_ c}} {{\lam \One \_ d}}}
      \App x}
    }
    \App x
  }
  \App x
\end{equation*}
that exhibits the desired computational behaviour.

Using this encoding, we can express the levitation rules without use of \En:
\begin{mathpar}
\inferrule* {
  \Gamma \ENT {}
} {
  \Gamma \ENT
    \Desc
  \EQ {
    \fix {
      \left(\uSigma
        {\underline {\type 4}}
        {\left(\lam {\underline {\type 4}} x \underline {\definition{switch4}} \App
\uOne \App
(\uSigma \Set {(\lam \Set S \uHind {\name S} \uOne)}) \App
(\uInd \uOne) \App
(\uSigma {\Set} {(\lam \Set \_ \uInd \uOne)}) \App
{\name x}\right)}
      \right)
    }
  } \OF \Set
}
\end{mathpar}
where 
\begin{align*}
\uOne &:= \con \App \cons {\underline{\term 0}} \nil\\
\uSigma \App S \App D &:= \con \App \cons {\underline{\term 1}} {\cons S  {\cons D \nil}}\\
\uInd \App D &:= \con \App \cons {\underline{\term 2}} {\cons D \nil}\\
\uHind \App H \App D &:= \con \App \cons {\underline{\term 3}} {\cons H  {\cons D \nil}}
\end{align*}

\section{Can we levitate \One?}

Instead of having a built-in unit type, can we introduce it using levitation
just like we did with \Desc? In other words, can we replace the introduction
rule of {\One} by a computation rule
$\Gamma \ENT \One \EQ \fix \uOne \OF \Set$?

Using this alternative rule, let us try to construct its (hopefully) unique
inhabitant using the ${\fix}$-introduction rule
\begin{mathpar}
\inferrule* {
  \Gamma \ENT d \OF \interp \uOne {(\fix \uOne)} \EQ \fix \uOne
} {
  \Gamma \ENT \con \App d \OF \fix \uOne
}
\end{mathpar}

which tells us that any inhabitant must be of shape $\con \, d$, where $d$ is
again of type $\fix \uOne$. We have hit an infinite regress, meaning that no
inhabitant can be constructed.

\end{document}

%%% Local Variables:
%%% mode: latex
%%% TeX-master: t
%%% End:
